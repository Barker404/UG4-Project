% to choose your degree
% please un-comment just one of the following
\documentclass[bsc,frontabs,twoside,singlespacing,parskip,deptreport]{infthesis}     % for BSc, BEng etc.
% \documentclass[minf,frontabs,twoside,singlespacing,parskip,deptreport]{infthesis}  % for MInf

\begin{document}

\title{Targeted Influence in Social Networks}

\author{Lewis Barker}

% to choose your course
% please un-comment just one of the following
%\course{Artificial Intelligence and Computer Science}
%\course{Artificial Intelligence and Software Engineering}
%\course{Artificial Intelligence and Mathematics}
%\course{Artificial Intelligence and Psychology }   
%\course{Artificial Intelligence with Psychology }   
%\course{Linguistics and Artificial Intelligence}    
\course{Computer Science}
%\course{Software Engineering}
%\course{Computer Science and Electronics}    
%\course{Electronics and Software Engineering}    
%\course{Computer Science and Management Science}    
%\course{Computer Science and Mathematics}
%\course{Computer Science and Physics}  
%\course{Computer Science and Statistics}    

% to choose your report type
% please un-comment just one of the following
%\project{Undergraduate Dissertation} % CS&E, E&SE, AI&L
%\project{Undergraduate Thesis} % AI%Psy
\project{4th Year Project Interim Report}

\date{\today}

\abstract{
Abstract goes here.
}

\maketitle

\section*{Acknowledgements}
Acknowledgements go here. 

\tableofcontents

%\pagenumbering{arabic}


\chapter{Introduction}


\chapter{Background}
\section{Concept}
ToDo
\section{Related Works}
ToDo

\chapter{Work Done and Results}
\section{Formal Definitions}

\subsection{Input}
Initially we are given a \textbf{graph} $G = (V, E \subseteq V \times V)$ and $k$ \textbf{messages} $ m_{1} ... m_{k}$, which have \textbf{sources} $s_{1} ... s_{k}$ and \textbf{destinations} $d_{1} ... d_{k}$. 

The graph $G$ represents the social network, with vertices in $V$ being the users of the network and edges in $E$ being connections between users within the network. A connection between two users $a$ and $b$ means in our context that user $a$ may see some content shared by user $b$ - however whether they actually see it or not is decided by the social network's algorithm.

\subsection{Rounds}
The system progresses in a series of \textbf{rounds}, representing periods of time passing in which messages can be passed on to other users. This allows us to more easily reason about the flow of information and also simulate the process.

For any round $t$, and for some vertex $v \in V$, the \textbf{shared set} $S_{t}(v)$ is the set of messages which were shared by $v$ in that round.

We assume that the source of a message will invariably share it initially (otherwise it has no chance to propagate). Therefore initially, at round 0, for each $v \in V$ the shared set is defined as the set of messages for which $v$ is the source:

\begin{equation}
S_{0}(v) = \{m_{x} \; | \; x \in [0, k] \wedge s_{x} = v\}
\end{equation}


We also define, for each $v \in V$ at any round $t > 0$, the \textbf{possible set} as all of the messages shared by neighbours of $v$ in the previous round - all the messages which $v$ may be able to see at this point:

\begin{equation}
P_{t}(v) = \bigcup_{u \in N(v)} \quad \bigcup_{m \in S_{t-1}(u)} (u, m)
\end{equation}

In the above equation, $N(v)$ is the set of neighbouring vertices of v, and $(u, m)$ is a 2-tuple. Given a tuple $p = (u, m)$, we say that $p_{(1)} = u$ and $p_{(2)} = m$, as a way of accessing the parts of the tuple.

Which of these messages are actually shown to $v$ is determined by the social network's algorithm - which is what we wish to design. We define the \textbf{shown set} of a user $v$ at some round $t$, $T_{t}(v)$, as the result of this algorithm - this will be a subset of the possible set for that user and round:

\begin{equation}
T_{t}(v) \subseteq P_{t}(v)
\end{equation}

Finally, from the messages that are shown to a user, the user will share some of them. This is decided by the user model algorithm. The result of this form the \textbf{shared set} of that user for this round - all messages in the user's shared set must have been part of their shown set for that round:

\begin{equation}
\forall m \in S_{t}(v) . \exists (u, n) \in T_{t}(v) . m = n
\end{equation}

This is then used as the basis for the possible set of neighbouring vertices in the next round. With this, we need only define the sharing and user model algorithms to be able to simulate the spread of messages throughout the network.


\section{Network Graph}


\section{Simulation Program}
ToDo

\section{User Models}
ToDo
\section{Showing Models With Results}
ToDo

\chapter{Future Plan}
\section{Further Work}
ToDo
\section{Plan for Semester 2}
ToDo
\chapter{Conclusion}
ToDo
% use the following and \cite{} as above if you use BibTeX
% otherwise generate bibtem entries
\bibliographystyle{plain}
\bibliography{mybibfile}


\end{document}
