% to choose your degree
% please un-comment just one of the following
\documentclass[bsc,frontabs,twoside,singlespacing,parskip,deptreport]{infthesis}     % for BSc, BEng etc.
% \documentclass[minf,frontabs,twoside,singlespacing,parskip,deptreport]{infthesis}  % for MInf

\usepackage{hyperref}
\usepackage{cite}

\begin{document}

\title{Targeted Influence in Social Networks}

\author{Lewis Barker}

% to choose your course
% please un-comment just one of the following
%\course{Artificial Intelligence and Computer Science}
%\course{Artificial Intelligence and Software Engineering}
%\course{Artificial Intelligence and Mathematics}
%\course{Artificial Intelligence and Psychology }   
%\course{Artificial Intelligence with Psychology }   
%\course{Linguistics and Artificial Intelligence}    
\course{Computer Science}
%\course{Software Engineering}
%\course{Computer Science and Electronics}    
%\course{Electronics and Software Engineering}    
%\course{Computer Science and Management Science}    
%\course{Computer Science and Mathematics}
%\course{Computer Science and Physics}  
%\course{Computer Science and Statistics}    

% to choose your report type
% please un-comment just one of the following
%\project{Undergraduate Dissertation} % CS&E, E&SE, AI&L
%\project{Undergraduate Thesis} % AI%Psy
\project{4th Year Project Interim Report}

\date{\today}

\abstract{
Abstract goes here.
}

\maketitle

\section*{Acknowledgements}
Acknowledgements go here. 

\tableofcontents

%\pagenumbering{arabic}


\chapter{Introduction}


\chapter{Background}
\section{Concept}
ToDo
\section{Related Works}
ToDo

\chapter{Work Done and Results}
\section{Formal Definitions}

\subsection{Input}
Initially we are given a \textbf{graph} $G = (V, E \subseteq V \times V)$ and $k$ \textbf{messages} $ m_{1} ... m_{k}$, which have \textbf{sources} $s_{1} ... s_{k}$ and \textbf{destinations} $d_{1} ... d_{k}$. 

The graph $G$ represents the social network, with vertices in $V$ being the users of the network and edges in $E$ being connections between users within the network. A connection between two users $a$ and $b$ means in our context that user $a$ may see some content shared by user $b$ - however whether they actually see it or not is decided by the social network's algorithm.

\subsection{Rounds}
The system progresses in a series of \textbf{rounds}, representing periods of time passing in which messages can be passed on to other users. This allows us to more easily reason about the flow of information and also simulate the process.

For any round $t$, and for some vertex $v \in V$, the \textbf{shared set} $S_{t}(v)$ is the set of messages which were shared by $v$ in that round.

We assume that the source of a message will invariably share it initially (otherwise it has no chance to propagate). Therefore initially, at round 0, for each $v \in V$ the shared set is defined as the set of messages for which $v$ is the source:

\begin{equation}
S_{0}(v) = \{m_{x} \; | \; x \in [0, k] \wedge s_{x} = v\}
\end{equation}


We also define, for each $v \in V$ at any round $t > 0$, the \textbf{possible set} as all of the messages shared by neighbours of $v$ in the previous round - all the messages which $v$ may be able to see at this point:

\begin{equation}
P_{t}(v) = \bigcup_{u \in N(v)} \quad \bigcup_{m \in S_{t-1}(u)} (u, m)
\end{equation}

In the above equation, $N(v)$ is the set of neighbouring vertices of v, and $(u, m)$ is a 2-tuple. Given a tuple $p = (u, m)$, we say that $p_{(1)} = u$ and $p_{(2)} = m$, as a way of accessing the parts of the tuple.

Which of these messages are actually shown to $v$ is determined by the social network's algorithm - which is what we wish to design. We define the \textbf{shown set} of a user $v$ at some round $t$, $T_{t}(v)$, as the result of this algorithm - this will be a subset of the possible set for that user and round:

\begin{equation}
T_{t}(v) \subseteq P_{t}(v)
\end{equation}

Finally, from the messages that are shown to a user, the user will share some of them. This is decided by the user model algorithm. The result of this form the \textbf{shared set} of that user for this round - all messages in the user's shared set must have been part of their shown set for that round:

\begin{equation}
\forall m \in S_{t}(v) . \exists (u, n) \in T_{t}(v) . m = n
\end{equation}

This is then used as the basis for the possible set of neighbouring vertices in the next round. With this, we need only define the showing and user model algorithms to be able to simulate the spread of messages throughout the network.

\section{Simulation Program}
For the purpose of running simulations of message spread in networks, a program was written allowing for running various simulations and receiving relevant results. The program was written in Python, using the NetworkX library [reference?] to represent and manipulate the network graphs. The program is split into classes, many of which have abstract base classes, to allow for easily creating different versions to swap in and out.

There is a \texttt{GraphGenerator} abstract base class, subclasses of which encapsulate the knowledge specific to the graph model being used - including how to create a graph of that type, and how the nodes should be positioned in the visualisation (for example if the network is based on a grid layout, it should be positioned as such). The rest of the program is totally independent from this knowledge, meaning that by simply using a different subclass the graph generation method can be modified completed - for example the graph could be loaded from a static file.

The showing algorithms are contained within subclasses of the \texttt{ShowModel} class, the main method of which is \texttt{show\_alg} which given a user's possible set and some additional information about the state of the network, returns a subset of the messages to be shown to them. Similarly, the user models algorithms are represented by subclasses of the \texttt{ShareModel} class, which has a method \texttt{share\_alg} to take the shown set and return the messages which the user will share. Changing which of these subclasses is used in the simulation allows comparing the performance of different algorithms.

These classes are brought together by the \texttt{Simulation} class, which is given an instance of each of \texttt{GraphGenerator}, \texttt{ShowModel}, and \texttt{ShareModel} on creation. The simulation can then be either run once or repeated multiple times, collating the results. When repeating the simulation, it can set to either use the exact same network graph or to regenerate the graph using the same generation method - in the case of randomly generated graphs, this allows for repeating the simulation to remove random differences caused by specific graphs. The simulation class can also create visualisations as outputs. It can output either images of the state at each round of the simulation, or a single video of the entire simulation. In these visualisations, a single message is highlighted as it spreads through the network, while the other vertices are coloured based on their level of traffic. This allows for seeing how a message moves through the graph and how far it spreads, as well as where potential bottlenecks occur and how busy the network is as a whole.

\section{Network Graph}
An important factor in how a message showing algorithm performs is the network that it is being performed on. If the network has invalid properties, the a successful algorithm may not be successful on other networks.

For the majority of this project, Kleinberg's model for small world graphs was used\cite{Kleinberg00}. This is a model which randomly creates a graph that fits certain properties. Initially, nodes are arranged in a grid layout, with a "grid distance" from each other. Each node is connected to its immediate neighbours (those a grid distance of 1 away). Each node is then connected to a constant number $q$ of other nodes. The probability of connecting node $u$ to node $v$ in this way is proportional to $d(u, v)^{-r}$ where $d(u, v)$ is the grid distance between $u$ and $v$, and $r$ is a constant affecting how likely the links are to be "far-reaching". If $q$ is 0, then the node will be linked to other nodes uniformly. If $q$ is high, the long links are more likely to connect closer nodes.

[diagram of Kleinberg's model]

This model provides several properties which make it suitable to use in place of a social network. Since it is randomly generated, we can run experiments using multiple different graphs to avoid results being skewed by peculiarities of individual networks. 

Additionally it fits into a category of graphs know as "Small World" graphs. This is related to Milgram's famous "Six Degrees of Separation" experiment\cite{Milgram67,TraversMilgram69}, in which he randomly chose individuals and asked them to forward a message on to a certain "target" via people they knew. Of those messages that reached the target, the median number of "steps" required was 6, displaying the existence of short paths within social networks - a fact that has also been seen in other studies\cite{MilgramBackup1,MilgramBackup2}. In addition to these paths existing, the experiment showed that it was possible for the intermediate individuals to find these paths without knowing the full structure of the network - they were able to conduct a distributed search. 

Kleinberg's model replicates both of these properties. For $q \ge 1$ the expected diameter of a graph generated using this model is $\theta (\log n)$ and a route between two nodes can be found in a distributed manner using $\theta (\log^{2}n)$ steps\cite{AnalyzingKleinberg}. This is achieved by sending the message as close as possible to the destination (in terms of grid distance) at each step.

This can be related to real networks intuitively. Individuals are more likely to be connected to those "close" to them - which in turn forms clusters of connections - but there will also be some "long-distance" connections that can connect clusters. To send a message to a far away target, a reasonable strategy would be to send to the person you know who is "closest" to the target - they are more likely to be connected to them.

While Kleinberg's small world model has been used primarily up to this point, other graph types may provide additional interesting insights. For example, using a plain grid could provide a baseline for what can be expected, to see if the algorithm developed takes advantage of the long link properties of the network. Alternatively, an real social network dataset could be used to see if the algorithm also work in actual networks.

\section{User Models}
While the main focus of this project was on the effect of the social network's "showing" algorithm, the model used also requires an algorithm to represent the user's actions. This algorithm should decide which of the messages shown to a user they then share - it chooses $S_{t}(v)$ based on $T_{t}(v)$. Depending on how this decision is modelled, it can add an element of uncertainty or randomness to the system, affecting which "showing" algorithms perform well.

\subsection{Basic User Model}
Initially, a very basic random selection method was used to choose the $S_{t}(v)$. Firstly, the user only considers the top $a$ of the messages shown to them (or all of the messages if there is less than $a$), where $a$ is a global constant - this represents the user's "attention", being unable to look at every message shown to them [expand on this?][reference attention paper?]. In the implementations used, the value of $a$ is also known to the showing algorithm, which does not show more than this many messages. From these $a$ messages, $b$ are selected at random and shared (or all of them are shared if there are less than $b$), where $b$ is a global constant and $b \le a$.

This provides us with a degree of randomness - if there are more than $b$ messages shown to the user, we can't tell which ones will be passed on. However in the case where there are less than or equal to $b$ messages shown to the user, this model acts unrealistically - the user will share every message they see. A real social network user would likely be more predictable (unless there was some incentive to share messages), and may not share all of the messages in nay circumstances. As a result of this unnatural behaviour, showing each message to only a single user at each step was found to be a highly effective strategy - however in a real situation this would likely result in losing the message at an uncooperative user.

\subsection{Probabilistic User Model}
A more realistic user model can be created using more probabilistic methods. In this user model, the concept of a maximum attention of $a$ is retained, with the user considering up to $a$ messages. However from these $a$ messages, rather than choose a set amount at random each message is shared with a probability $p$, where $p$ is a global constant.

This results in a situation where a user may share all the messages they see, or may share none. This emulates how a user will only share certain messages based on some criteria unknown to the network. With this model, if messages are not sent to enough users then they are likely to not be shared and be lost.

\section{Showing Models With Results}
ToDo

\chapter{Future Plan}
\section{Further Work}
Analysis:
- Network Flows
- Multicommodity

Experimentation:
- Other algorithms
- Diffuse distance
- other graph types (eg real network)


\section{Plan for Semester 2}
ToDo
\chapter{Conclusion}
ToDo

% use the following and \cite{} as above if you use BibTeX
% otherwise generate bibtem entries
\bibliographystyle{plain}
\bibliography{interimBib}


\end{document}
