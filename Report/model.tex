\documentclass{article}

\setlength{\parindent}{0em}
\setlength{\parskip}{1em}

\begin{document}
\author{Lewis Barker}
\title{Project model}
\date{November 2015}
\maketitle

\section{Input}
We are given a \textbf{graph} $G = (V, E \subseteq V \times V)$ and $k$ \textbf{messages} $ m_{1} ... m_{k}$, which have \textbf{sources} $s_{1} ... s_{k}$ and \textbf{destinations} $d_{1} ... d_{k}$.

\section{Rounds}
The system progresses in a series of \textbf{rounds}. For any round $t$, and for some node $v \in V$, the \textbf{shared set} $S_{t}(v)$ is the set of messages which were shared by $v$ in that round.

Initially, at round 0, for each $v \in V$ the shared set is defined as:

\begin{equation}
S_{0}(v) = \{m_{x} ~|~ x \in [0, k] \wedge s_{x} = v\}
\end{equation}

That is, the set of messages for which $v$ is the source.

We also define, for each $v \in V$ at any round $t > 0$, the \textbf{possible set} as follows:

\begin{equation}
P_{t}(v) = \bigcup_{u \in N(v)} \quad \bigcup_{m \in S_{t-1}(u)} (u, m)
\end{equation}

Where $N(v)$ is the set of neighbouring nodes of v, and $(u, m)$ is a 2-tuple. Given a tuple $p = (u, m)$, we say that $p_{(1)} = u$ and $p_{(2)} = m$, as a way of accessing the parts of the tuple.

The possible set of $v$ therefore consists of all of the messages shared by neighbours of $v$ in the previous round - all the messages which $v$ can see at this point.

\end{document}